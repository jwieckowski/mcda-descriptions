
The VIseKriterijumska Optimizacija I Kompromisno Resenje (VIKOR) method was introduced by Opricovic. It is characterized by the approach that produces three distinctive rankings ($S$, $R$, and compromised solution $Q$). The standard technique that operates on the crisp numbers was extended with many fuzzy extensions that allow for calculating results when uncertain data appear. One method that is eagerly applied in fuzzy problems is Triangular Fuzzy Number VIKOR. Main steps of this technique are presented below. \\ 

\noindent \textbf{Step 1.} Determination of the triangular fuzzy decision matrix, which contains $m$ alternatives and $n$ criteria ($i = 1, 2, \ldots, m$ and $j = 1, 2, \ldots, n$) (\ref{eq:vikor1}):

\begin{equation}
X=\left[\begin{array}{ccccc}
x_{11} & \cdots & x_{1 j} & \cdots & x_{1 n} \\
\vdots & \ddots & \vdots & \ddots & \vdots \\
x_{i 1} & \cdots & x_{i j} & \cdots & x_{i n} \\
\vdots & \ddots & \vdots & \ddots & \vdots \\
x_{m 1} & \cdots & x_{m j} & \cdots & x_{m m}
\end{array}\right]
\label{eq:vikor1}
\end{equation}

\noindent where $x_{ij}$ is represented as Triangular Fuzzy Number ($x^L, x^M, x^U$). \\


\noindent \textbf{Step 2.} Determination of the Ideal and Non-Ideal Solution for each criterion based on the criterion type. \\

\noindent For the profit type criterion (\ref{eq:vikor2}):

\begin{equation}
    \begin{aligned}
    & \tilde{f}_{i}^{*}=M A X \tilde{f}_{i j}
    & \tilde{f}_{i}^{\circ}=M I N \tilde{f}_{i j}
    \end{aligned}
    \label{eq:vikor2}
\end{equation}

\noindent and for the cost type criterion (\ref{eq:vikor3}):

\begin{equation}
\begin{aligned}
    & \tilde{f}_{i}^{*}=M I N \tilde{f}_{i j}
    & \tilde{f}_{i}^{\circ}=M A X \tilde{f}_{i j}
\end{aligned}
    \label{eq:vikor3}
\end{equation}

\noindent \textbf{Step 3.} Calculation of the normalized fuzzy difference from Ideal Solution. \\

\noindent For the profit type criterion (\ref{eq:vikor4}):

\begin{equation}
    \tilde{D}_{i j}=\left(\tilde{f}_{i}^{*} \Theta \tilde{f}_{i j}\right) /\left(x_{i}^{*U}-x_{i}^{\circ L}\right)
    \label{eq:vikor4}
\end{equation}

\noindent and for the cost type criterion (\ref{eq:vikor5}):

\begin{equation}
     \tilde{D}_{i j}=\left(\tilde{f}_{i j} \Theta \tilde{f}_{i}^{*}\right) /\left(x_{i}^{\circ U}-x_{i}^{* L}\right)
     \label{eq:vikor5}
\end{equation}

\noindent \textbf{Step 3.} Calculation of the fuzzy values of $\tilde{S}$ and $\tilde{R}$ (\ref{eq:vikor6}).

\begin{equation}
\begin{aligned}
& \tilde{S}_{j}=\sum_{i=1}^{n}{ }_{\oplus}\left(\tilde{w}_{i} \otimes \tilde{D}_{i j}\right) 
& \tilde{R}_{j}=\operatorname{MAX}_{i}\left(\tilde{w}_{i} \otimes \tilde{D}_{i j}\right)
\end{aligned}
\label{eq:vikor6}
\end{equation}

where $\tilde{w}_{i}$ is weight for the \textit{i-th} criterion.\\

\noindent \textbf{Step 4.} Determination of the fuzzy $\tilde{Q}$ values (\ref{eq:vikor7}).

\begin{equation}
    \tilde{Q}_{j}=v\frac{\left(\tilde{S}_{j} \Theta \tilde{S}^{*}\right)}{\left(S^{\circ U}-S^{* L}\right)} \oplus \frac{(1-v)\left(\tilde{R}_{j} \Theta \tilde{R}^{*}\right)}{\left(R^{\circ U}-R^{* L}\right)}
    \label{eq:vikor7}
\end{equation}

\noindent where $v$ can be adjusted by expert ($v$ $\in$ [0, 1]), and 

\begin{equation}
\begin{aligned}
& \tilde{S}^{*L}=MIN(\tilde{S}_{j}^L) & S^{\circ U}=MAX(S_{j}^{U}) \\
& \tilde{R}^{*L}=MIN(\tilde{R}_{j}^L) & R^{\circ U}=MAX(R_{j}^{U})  
\end{aligned}
\label{eq:vikor8}
\end{equation}

\noindent \textbf{Step 5.} Calculation of the defuzzified preference values (\ref{eq:vikor9}).

\begin{equation}
\tilde{N}_i=\frac{(2 x_{j}^M+x_{j}^L+x_{j}^U)}{4}
\label{eq:vikor9}
\end{equation}