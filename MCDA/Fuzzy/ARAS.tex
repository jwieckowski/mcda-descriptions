
The fuzzy extension of the Additive Ratio Assessment (ARAS) method was presented by Zavadskas et al. It allows for operating in an uncertain environment based on the data represented as TFNs. The main steps of the fuzzy ARAS procedure can be described as follows. \\

\noindent \textbf{Step 1.} Determination of the triangular fuzzy decision matrix, which contains $m$ alternatives and $n$ criteria ($i = 1, 2, \ldots, m$ and $j = 1, 2, \ldots, n$) (\ref{eq:aras1}).

\begin{equation}
X=\left[\begin{array}{ccccc}
x_{11} & \cdots & \tilde{x}_{1 j} & \cdots & x_{1 n} \\
\vdots & \ddots & \vdots & \ddots & \vdots \\
x_{i 1} & \cdots & x_{i j} & \cdots & x_{i n} \\
\vdots & \ddots & \vdots & \ddots & \vdots \\
x_{m 1} & \cdots & x_{m j} & \cdots & x_{m m}
\end{array}\right]
\label{eq:aras1}
\end{equation}

\noindent where $x_{ij}$ is represented as Triangular Fuzzy Number ($x^L, x^M, x^U$). \\

\noindent \textbf{Step 2.} Determination of the optimal value of each criterion value (\ref{eq:aras2}).

\begin{equation}
\tilde{X}=\left[\begin{array}{ccccc}
\tilde{x}_{01} & \cdots & \tilde{x}_{0 j} & \cdots & \tilde{x}_{0 n} \\
\vdots & \ddots & \vdots & \ddots & \vdots \\
\tilde{x}_{i 1} & \cdots & \tilde{x}_{i j} & \cdots & \tilde{x}_{i n} \\
\vdots & \ddots & \vdots & \ddots & \vdots \\
\tilde{x}_{m 1} & \cdots & \tilde{x}_{m j} & \cdots & \tilde{x}_{m m}
\end{array}\right]
\label{eq:aras2}
\end{equation}

\noindent where $\tilde{x}_{0j}$ denotes the optimal value of $j$ criterion (for a profit criteria $\tilde{x}_{0j}$ = $\underset{i}{max}$ $x_{ij}$; for a cost criteria $\tilde{x}_{0j}$ = $\underset{i}{min}$ $x_{ij}$). \\

\noindent \textbf{Step 3.} Calculation of the normalized fuzzy decision matrix (\ref{eq:aras3}).

\begin{equation}
\tilde{\bar{X}}=\left[\begin{array}{ccccc}
\tilde{\bar{x}}_{01} & \cdots & \tilde{\bar{x}}_{0 j} & \cdots & \tilde{\bar{x}}_{0 n} \\
\vdots & \ddots & \vdots & \ddots & \vdots \\
\tilde{\bar{x}}_{i 1} & \cdots & \tilde{\bar{x}}_{i j} & \cdots & \tilde{\bar{x}}_{i n} \\
\vdots & \ddots & \vdots & \ddots & \vdots \\
\tilde{\bar{x}}_{m 1} & \cdots & \tilde{\bar{x}}_{m 1} & \cdots & \tilde{\bar{x}}_{m n}
\end{array}\right]
\label{eq:aras3}
\end{equation}

\noindent where for profit criteria, the normalization formula is presented as follows (\ref{eq:aras4}):

\begin{equation}
\tilde{\bar{x}}_{i j}=\frac{\tilde{x}_{i j}}{\sum_{i=0}^m \tilde{x}_{i j}},
\label{eq:aras4}
\end{equation}

\noindent and for the cost criteria, the normalization formula is presented as follows (\ref{eq:aras5})

\begin{equation}
\tilde{\bar{x}}_{i j}=\frac{\frac{1}{\tilde{x}_{i j}}}{\sum_{i=0}^m \frac{1}{\tilde{x}_{i j}}}
\label{eq:aras5}
\end{equation}

\noindent \textbf{Step 4.} Calculation of the weighted normalized fuzzy decision matrix with the formula (\ref{eq:aras6}):

\begin{equation}
\widetilde{\widehat{X}}=\left[\begin{array}{ccccc}
\widetilde{\widehat{x}}_{01} & \cdots & \widetilde{\widehat{x}}_{0 j} & \cdots & \widetilde{\widehat{x}}_{0 n} \\
\vdots & \ddots & \vdots & \ddots & \vdots \\
\widetilde{\widehat{x}}_{i 1} & \cdots & \widetilde{\widehat{x}}_{i j} & \cdots & \widetilde{\widehat{x}}_{i n} \\
\vdots & \ddots & \vdots & \ddots & \vdots \\
\widetilde{\widehat{x}}_{m 1} & \cdots & \widetilde{\widehat{x}}_{m j} & \cdots & \widetilde{\widehat{x}}_{m m}
\end{array}\right]
\label{eq:aras6}
\end{equation}

\noindent where $\widetilde{\widehat{x}}_{i j}=\tilde{\bar{x}}_{i j} \times \widetilde{w}_j, \quad i=0,1, \ldots, m, j=1,2, \ldots, n$. \\

\noindent \textbf{Step 5.} Determination of the overall performance index for each alternative (\ref{eq:aras7}).

\begin{equation}
\widetilde{S}_{i}= \sum_{j=1}^{n} \widetilde{\widehat{x}}_{i j}, \quad i=0, 1, \ldots, m
\label{eq:aras7}
\end{equation}

\noindent \textbf{Step 6.} Calculation of the defuzzified values of performance index (\ref{eq:aras8}).

\begin{equation}
\widetilde{S}_i=\frac{1}{3}\left(\widetilde{S}_{i}^L+\widetilde{S}_{i}^M+\widetilde{S}_{i}^U\right), \quad i=0,1, \ldots, m
\label{eq:aras8}
\end{equation}

\noindent \textbf{Step 7.} Determination of the utility degree of each alternative with the formula (\ref{eq:aras9}):

\begin{equation}
Q_i=\frac{S_i}{S_0}, \quad i=0,1, \ldots, m
\label{eq:aras9}
\end{equation}