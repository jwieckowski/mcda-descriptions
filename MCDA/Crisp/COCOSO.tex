\noindent \textbf{Step 1.} Definition of a decision matrix of dimension $n \times m$, where $n$ is the number of alternatives, and $m$ is the number of criteria (\ref{equ:mat}).

\begin{equation}
\label{equ:mat}
x_{i j}=\left[\begin{array}{llll}
x_{11} & x_{12} & \ldots & x_{1 m} \\
x_{21} & x_{22} & \ldots & x_{2 m} \\
\ldots & \ldots & \ldots & \ldots \\
x_{n 1} & x_{n 2} & \ldots & x_{n m}
\end{array}\right]
\end{equation}



\noindent \textbf{Step 2.} Normalization the decision matrix, where for profit criteria use the equation~(\ref{equ:profit}), and for cost, criteria use the equation~(\ref{equ:cost}). In this study, The Minimum-Maximum normalization method was used.

\begin{equation}
\label{equ:profit}
    r_{ij} = \frac{x_{ij} - \min_{i}{x_{ij}}}{\max_{i}{x_{ij}} - \min_{i}{x_{ij}}}
\end{equation}

\begin{equation}
\label{equ:cost}
    r_{ij} = \frac{\max_{i}{x_{ij}} - x_{ij}}{\max_{i}{x_{ij}} - \min_{i}{x_{ij}}}
\end{equation}


\noindent \textbf{Step 3.} Calculation of the weighted sum of the comparison sequence and the total power weight of the comparison sequences for each alternative. The values of $S_i$ are based on the grey relationship generation method (\ref{equ:SI}), and for $P_i$ the values are achieved according to the multiplicative WASPAS setting (\ref{equ:PI}).


\begin{equation}
\label{equ:SI}
    S_i = \sum_{j=1}^{n} (w_j r_{ij})
\end{equation}

\begin{equation}
\label{equ:PI}
    P_i = \sum_{j=1}^{n} (r_{ij})^{w_j}
\end{equation}


\noindent \textbf{Step 4.} Computation of the relative weights of alternatives using aggregation strategies. The formulas determine the strategies (\ref{equ:s1})-(\ref{equ:s3}), where the first strategy expresses the average of the sums of WSM and WPM scores (\ref{equ:s1}), the second strategy expresses the sum of WSM and WPM scores over the best (\ref{equ:s2}), and the third strategy expresses the compromise strategy of WSM and WPM by using the $\lambda$ value (\ref{equ:s3}). In this study, a $\lambda$ value of 0.5 was used.

\begin{equation}
\label{equ:s1}
k_{i a}=\frac{P_{i}+S_{i}}{\sum_{i=1}^{m}\left(P_{i}+S_{i}\right)}
\end{equation}

\begin{equation}
\label{equ:s2}
k_{i b}=\frac{S_{i}}{\min _{i} S_{i}}+\frac{P_{i}}{\min _{i} P_{i}}
\end{equation}

\begin{equation}
\label{equ:s3}
k_{i c}=\frac{\lambda\left(S_{i}\right)+(1-\lambda)\left(P_{i}\right)}{\left(\lambda \max _{i} S_{i}+(1-\lambda) \max _{i} P_{i}\right)} ; \quad 0 \leqslant \lambda \leqslant 1
\end{equation}

\noindent \textbf{Step 5.} Establish the final ranking of alternatives based on $k_i$ values defined using the formula (\ref{equ:ki}). The higher the $k_i$ value, the higher the ranking.

\begin{equation}
\label{equ:ki}
k_{i}=\left(k_{i a} k_{i b} k_{i c}\right)^{\frac{1}{3}}+\frac{1}{3}\left(k_{i a}+k_{i b}+k_{i c}\right)
\end{equation}
