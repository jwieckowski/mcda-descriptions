The pairwise comparison method is a very old technique of ordering criteria and involves comparing each criterion against every other criterion in pairs. It forces the decision-maker to give thorough consideration to all elements of a decision problem. The number of comparisons can be determined by:

\begin{equation}
    o = \frac{m(m-1)}{2}
\end{equation}

where:
\begin{itemize}
    \item o = the number of comparisons; and
    \item m = the number of criteria
\end{itemize}

The calculation of weighting vector includes three main steps. \\

\textbf{Step 1 -  Development of a matrix comparing the criteria.} The intensity values are used to fill in the matrix of comparisons. What is important, that not all the values from given range need to be used. For example, values 1, 3, 5, 7, 9 or 1, 5, 9 could be used in the same way, it depends if the decision-maker finds it difficult to distinguish between definitions. Diagonal of the matrix is always filled with 1, when the upper right values are filled in based on the comparisons and the lower left are inverse values because of the assumption that the matrix is reciprocal. \\

\textbf{Step 2 - Calculation of the criteria weights.} It is done by summing the values in each column, dividing each element by the column total, and dividing the sum of the normalized scores for each row by the number of criteria. \\

\textbf{Step 3 - Calculation a consistency ratio.} If the consistency ratio is less than 0.10, then the ratio indicates a reasonable level of consistency in the pairwise comparisons. If it is larger than 0.10, the values of the ratio are indicative of inconsistent judgments. \\

This method is often criticized for asking for the relative importance of evaluation criteria without reference to the scales on which the criteria are measured. The occurring fuzziness may mean that decision-makers interpret the questions in a different and possibly wrong manner.
