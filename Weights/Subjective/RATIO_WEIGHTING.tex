The ratio method requires the decision-maker to rank the relevant criteria according to their importance, similarly to Ranking method. However, significant difference can be noticed \cite{odu2019weighting}. The least important criterion should be assigned to a value of $10$ and the rest of them should be judged as multiplies of $10$. The resulting values are then normalized to sum to one. The Ratio Weighting method is presented as an algebraic, decomposed, direct procedure \cite{zardari2015weighting}. On the other hand, it is not well described by the authors, if the greater discrepancies between assigned values will cause more diverse. The subjective techniques should handle different users demands, making it possible to apply by even a professional or beginner user. The criteria weights using Ratio Weighting technique can be calculated using (\ref{eq:rw}):

\begin{equation}
    W_i = \frac{\alpha_i}{\sum^{n}_{i=1} \alpha_i}
\label{eq:rw}
\end{equation}

where $n$ is the number of criteria, $\alpha_i$ is the assigned to $i-th$ criterion value $\in$ [10, 20, 30, 40, \ldots]. \\
