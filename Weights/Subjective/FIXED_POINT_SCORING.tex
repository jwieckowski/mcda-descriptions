The main rule in this method is that the decision-maker must distribute a fixed number of points ($100$ is treated as a base amount of points) amongst the considered criteria \cite{hajkowicz2000evaluation}. The higher point score, the greater importance of the significant factor. Percentages are a popular manner to use a measure because many decision-makers are familiar with this approach. The advantage of fixed-point scoring is that it forces decision-makers to make trade-offs in the decision problem \cite{odu2019weighting}. Through Fixed Point Scoring (FPS), it is only possible to ascribe higher importance to one criterion by lowering the importance of another. It presents a difficult task to the decision-maker, which requires careful consideration of the relative importance of each criterion. Fixed point scoring is the most direct means of obtaining weighting information from the decision-maker. Assumption of the FPS methods are the same as in the Point Allocation method \cite{zardari2015weighting}. Calculation of weights based on the expert assessment is performed by (\ref{eq:fps}):

\begin{equation}
    W_i=\frac{\alpha_{i}}{N}
    \label{eq:fps}
\end{equation}

where N is the set fixed number of points for the expert to allocate between the criteria set identified in the problem.
